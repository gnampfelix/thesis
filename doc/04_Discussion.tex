% !TEX root = thesis.tex
%%%%%%%%%%%%%%%%%%%%%%%%%%%%%%%%%%%%%%%%%%%%%%%%%%%%%%%%%%%%%%%%%%%%
% Discussion and Outlook
%%%%%%%%%%%%%%%%%%%%%%%%%%%%%%%%%%%%%%%%%%%%%%%%%%%%%%%%%%%%%%%%%%%%

\chapter{Discussion}
  \label{sec:diss}

The aim of this thesis is to explore some aspects of FracMinHash in the context
of phylogenies for \textit{Phytophthora} and to compare the results to
outlines calculated with Mash.

FracMinHash is generally able to reproduce trees and outlines that were
previously calculated using Mash distances based on genome sequences of
\textit{Phytophthora}. The phylogenies visualize well defined clusters of
species that are in line with the clades and subclades described in literature
\cite{abadPhytophthoraTaxonomicPhylogenetic2023a,yangExpandedPhylogenyGenus2017}.

There are some differences between the splits based on the FracMinHash distances
and the splits based on the Mash distances that also impact the interpretation
of the results slightly. However, those differences are not substantially,
especially when taking the differences between published phylogenies into
account.

With a dataset of closely related species (such as dataset A), the choice of
parameters $k$ and $s$ for sketching is not influencing the results
considerably: choice of $k \in \{19, 20, 21, 25\}$ or $s \in \{1000, 2000,
4000\}$ don't influcence clade clustering at all, only the placement of clades
differs to some degree that is again comparable to the differences between
FracMinHash and Mash.

The differences have impact, however, when including species with diverging
genome lengths or that are only distantly related, e.g. when including bacterial
or fungal genomes. Here, Mash isn't able to estimate evolutionary distances and
outputs a distance of $1$ more often than FracMinHash. Considering the
application of phylogenetic context, the user is required to give a distance
threshold for all genomes that should be part of the result outline. In the
experiments with dataset C, I have put all those genomes into the set of the
query genomes, which implies they will always be part of the resulting outline.
However, in a scenario where those are part of the reference database, the
failure of Mash to estimate distances below $1$, e.g. for \textit{Venturia
carpophila} would remove this species from the set (as setting a threshold of
$1$ would imply having no threshold at all).

Using a value of $s=2000$ for dataset C yields sketches for the highlighted
genomes with sizes in the interval $[2832, 44796]$. The sketch of size $44796$
belongs to \textit{Phytophthora infestans}, the sketch of size $2832$ to
\textit{Thermogemmatispora aurantia}. This reflects the different genome sizes
well, altough the sketches can become considerably larger compared to Mash
sketches. The argument that FracMinHash performs better because sketch sizes
are larger compared to Mash can be invalidaded by tuning the scaling parameter
$s$ such that relevant genomes approximately have a sketch size of $10000$. In
the experiments with dataset C, using values of $s=500$ (setting the bacterial
genome sketches to $\approx 10000$) and $s=3500$ (setting the fungal genomes
sketches to $\approx 10000$) didn't change the number of empty intersections of
the calculated sketches for the comparison of the fungal queries to
\textit{Phytophthora}. 

Moreover, scaling the sketch according to the genome size seems a sensible thing
to do to reflect the larger amount of data available for large genomes. Still,
expecting the user to supply a value for $s$ can feel arbitrary, similar to
$s_{mash}$ in Mash.

\textit{Phytophthora} were considered fungi during some period because they
share some aspects with fungal species. The phylogenetic outline generated based
on FracMinHash sketches in which fungal query genomes are placed between
\textit{Phytophthora}, albeit with some distance, shows this relationship.
Although the AAI values for those comparisons are at the lower end of the range
and the separation to the bacterial genomes is very small, the fact that the
fungal query sequences are more closely related to the \textit{Phytophthora}
reference sequences is already indicated here. The experiments with dataset C
thus support the hypothesis that FracMinHash is better suited for distance
estimation in those circumstances.

Altough the method is more sensitive for genomes with diverging genome sizes,
there is a lower boundary in terms of genome size. This observation was already
made in
\cite{irberLightweightCompositionalAnalysis2022,heraDebiasingFracMinHashDeriving2023}
and is repeated here. While it is impressive to see clusters that are in line
with clade membership for mtDNA datasets with sketch sizes not substantially
larger than $300$, the clade placement is not backed by literature
\cite{winkworthComparativeAnalysesComplete2022,abadPhytophthoraTaxonomicPhylogenetic2023a,
yangExpandedPhylogenyGenus2017}. This also exlcudes viral genomes from the set
of applicable sequences, as they tend to be even shorter \todo{cite} than mtDNA.

A general assumption of \textit{locality sensitive hashing} methods is that
hashes are evenly distributed across the genome. Using the coordinates of
$k$-mers inside a FracMinHash sketch, we could observe windows in some genomes
that contained more or less hashes that one would expect given the scaling
parameter $s$. This is true for 37 out of the 64 \textit{Phytophthora} reference
genomes. To ensure that this is result is not directly depending on a given hash
seed, the calculations were performed for 9 different hash seeds, using the
median count per window. From a practical point of view, having windows with
more hashes than expected does not necessarily imply that this window is
overrepresented in the sketch, i.e. the number of actual repeats was not
included. However, regions with lower counts than expected also exist and here,
we can assume that those windows are underrepresented.

The impact of this on the biological interpretation of the generated phylogenies
is probably minimal, as those differences can be linked to low sequence
complexity in those windows. To put it differently, those windows don't store
the same amount information as windows with high sequence complexity. For the
reference genome of \textit{Phytophthora infestans}, this is also backed by the
fact that there is not a single window with unusual hash counts that contains a
CDS. Because of this, a potential link to the two-speeds-genome hypothesis for
\textit{Phytophthora} - which is based repeat rich regions - could also not be
verified.

The genome size alone cannot explain the observation of those windows. While
there is a trend that genomes with a larger total amount of analysed windows
have windows with unusual counts, there are cases like \textit{Phytophthora
castanetorum} with a total number of $n=6928$ analysed windows out of which
$295$ had unusual counts. Compare this to \textit{Phytophthora cinnamomi} with a
total number of $n=10799$ analysed windows of which only $28$ have unusual
counts. Also, ny construction of the analysis, ambiguous bases cannot explain
the differences as all windows with a complexity score $C_m < 0$ where excluded. 

Computationally, it is interesting that the existing tools all use MurMur hash
to obtain hash values
\cite{ondovMashFastGenome2016,bagciMicrobialPhylogeneticContext2021,irberLightweightCompositionalAnalysis2022}.
In the last 10 years, many non cryptographic hash functions were published that
exceed MurMur hash performance, which in turn can be utilized for faster
runtime. It would be interesting to see if hash functions that are specialized
for hashing consecutive $k$-mers could be utilized to improve the performance in
that part of the programs even further \todo{find the hash function again}.

Altough the used hash functions are faster, the implementation of
\texttt{fmhdist} still only beats \texttt{sourmash} in terms of runtime when
exploting the fact that sourmash does not support multithreading out of the box.
When using a single core, both \texttt{sourmash} and \texttt{mash} outperform
\texttt{fmhdist} in terms of runtime. Without a deep dive into the architecture
of all involved implementations, one could only speculate about the reasons, the
choice of programming language for the heavy lifting (\texttt{rust} in the case
of \texttt{sourmash}, \texttt{c++} in the case of \texttt{mash}) could play a
role here. \todo{maybe memory footprint?}

Of course very important! You need to discuss the informatics as well as bio part of your thesis topic.\\\\
This section should include the following:
\begin{itemize}
  \item Short summary of the aim of the study
  \item Discussion of the results
  \item Contextualisation of the results/ Putting the results into context (that was set in the introduction)
  \item You can also discuss limitations of your thesis and formulate an outlook for future/ follow-up research (Outlook can become an extra chapter.)
  \item Finally, you can add a short conclusion of the main findings/ take aways of the thesis
\end{itemize}
Take your time for writing the discussion, besides the introduction chapter it is the most important chapter of your thesis.\\\\
Also do not subsection the discussion too heavily.\\\\
At least 5 pages.