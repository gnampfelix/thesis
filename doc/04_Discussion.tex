% !TEX root = thesis.tex
%%%%%%%%%%%%%%%%%%%%%%%%%%%%%%%%%%%%%%%%%%%%%%%%%%%%%%%%%%%%%%%%%%%%
% Discussion and Outlook
%%%%%%%%%%%%%%%%%%%%%%%%%%%%%%%%%%%%%%%%%%%%%%%%%%%%%%%%%%%%%%%%%%%%

\chapter{Discussion}
  \label{sec:diss}

The aim of this thesis is to explore FracMinHash in the context of phylogenies
for \textit{Phytophthora} and to compare the results to outlines calculated with
Mash.

FracMinHash is generally able to reproduce trees and outlines that were
previously calculated using Mash distances based on genome sequences of
\textit{Phytophthora}. The phylogenies visualize well defined clusters of
species that are in line with the clades and subclades described in literature
\cite{abadPhytophthoraTaxonomicPhylogenetic2023a,yangExpandedPhylogenyGenus2017}.

There are some differences between the splits based on FracMinHash distances and
the splits based on Mash distances that also impact the interpretation of the
results slightly. We see that some splits are only present in the Mash variant,
others only in the FracMinHash variant, the most notable being the difference
regarding the $S = \{1, 2, 3, 4, 5, 6, 7\}|\{8, 10\}$ split. We also see a
different total split weight. However, those differences are not substantially,
especially when taking the differences between published phylogenies into
account. 

With a dataset of closely related species (such as dataset \textbf{A}), the
choice of
parameters $k$ and $s$ for sketching is not influencing the results
considerably: choice of $k \in \{19, 20, 21, 25\}$ or $s \in \{1000, 2000,
4000\}$ don't influence clade clustering at all, only the placement of clades
differs to some degree that is again comparable to the differences between
FracMinHash and Mash.

The differences can matter, however, when including species with diverging
genome lengths or species that are only distantly related, e.g. when including
bacterial or fungal genomes. In this case, Mash isn't able to estimate
evolutionary distances and outputs a distance of $1$ more often than
FracMinHash. Considering the application of phylogenetic context, the user is
required to give a distance threshold for all genomes that should be part of the
result outline. In the experiments with dataset \textbf{C}, I have put all those
genomes into the set of the query genomes, which implies they will always be
part of the resulting outline. However, in a scenario where those are part of
the reference database, the failure of Mash to estimate distances below $1$,
e.g. for \textit{Venturia carpophila} would remove this species from the set (as
setting a threshold of $1$ would imply having no threshold at all).

Using a value of $s=2000$ for dataset \textbf{C} yields sketches for the
highlighted genomes with sizes in the interval $[2832, 44796]$. The sketch of
size $44796$ belongs to \textit{Phytophthora infestans}, the sketch of size
$2832$ to \textit{Thermogemmatispora aurantia}. This reflects the different
genome sizes well, altough the sketches can become considerably larger compared
to Mash sketches. The argument that FracMinHash performs better because sketch
sizes are larger compared to Mash can be invalidaded by tuning the scaling
parameter $s$ such that relevant genomes approximately have a sketch size of
$10000$. In the experiments with dataset \textbf{C}, using values of $s=500$
(setting the bacterial genome sketches to $\approx 10000$) and $s=3500$ (setting
the fungal genomes sketches to $\approx 10000$) didn't change the number of
empty intersections of the calculated sketches for the comparison of the fungal
queries to \textit{Phytophthora}. 

Moreover, scaling the sketch according to the genome size seems generally
reasonable to reflect the larger amount of data available for large genomes.
Still, expecting the user to supply a value for $s$ can feel arbitrary, similar
to $s_{mash}$ in Mash.

As described in Chapter~\ref{sec:background}, \textit{Phytophthora} were
considered fungi in the past because they share some aspects with true fungi
\cite{debaryResearchesNaturePotatofungus1876,rossmanWhyArePhytophthora2012}. The
phylogenetic outline based on FracMinHash sketches in which fungal query genomes
are placed between \textit{Phytophthora}, albeit with some distance, shows this
relationship, or at least avoids clustering the fungal queries together with the
bacterial queries. Although the AAI values for those comparisons are at the
lower end of the range and the separation to the bacterial genomes is very
small, the fact that the fungal query sequences are more closely related to the
\textit{Phytophthora} reference sequences than the bacterial queries is also
indicated here. The experiments with dataset \textbf{C} thus support the
hypothesis that FracMinHash is better suited for distance estimation in those
circumstances.

Altough the method is more sensitive for genomes with diverging genome sizes,
there is a lower boundary in terms of genome size. This observation was already
made in
\cite{irberLightweightCompositionalAnalysis2022,heraDerivingConfidenceIntervals2023}
and is repeated here. While it is impressive to see clusters that are in line
with clade membership for mtDNA datasets with sketch sizes not substantially
larger than $300$, the clade placement is not backed by literature
\cite{winkworthComparativeAnalysesComplete2022,abadPhytophthoraTaxonomicPhylogenetic2023a,
yangExpandedPhylogenyGenus2017}. This also exlcudes viral genomes from the set
of applicable sequences, as they tend to be even shorter than mtDNA
\cite{dimmockIntroductionModernVirology2001}.

A general assumption of \textit{locality sensitive hashing} methods is that
hashes are evenly distributed across the genome. Using the coordinates of
$k$-mers inside a FracMinHash sketch, we could observe windows in some genomes
that contained a different number of hashes that one would expect given the
scaling parameter $s$. This is true for 37 out of the 64 \textit{Phytophthora}
reference genomes. To ensure that this result is not directly depending on a
given hash seed, the calculations were performed for 9 different hash seeds,
using the median count per window. From a practical point of view, having
windows with more hashes than expected does not necessarily imply that this
window is overrepresented in the sketch, i.e. the number of actual repeats was
not included in the anlaysis. However, regions with lower counts than expected
also exist and here, we can assume that those windows are underrepresented.

The impact of this on the biological interpretation of the generated phylogenies
is probably minimal, as those differences can be linked to low sequence
complexity in those windows. To put it differently, those windows don't store
the same amount information as windows with high sequence complexity. This can
be explained intuitively: assume the hash value $h(i)$ for a $k$-mer $i$ is
smaller than $\frac{H}{s}$ and is therefore part of the FracMinHash sketch. Now,
every other instance of this $k$-mer has the same hash value and is thus also
represented in the window. If a $k$-mer is appearing repeatedly in a given
window of size $w$, we can assume this window is repeat rich and thus probably
has a low sequence complexity \cite{pirogovHighcomplexityRegionsMammalian2019}.
The same applies if $h(i)$ is greater than $\frac{H}{s}$.

For the reference genome of \textit{Phytophthora infestans}, there is not a
single window with unusual hash counts that contains a CDS. Because of this, a
direct link to the two-speeds-genome hypothesis for \textit{Phytophthora} -
which is based repeat rich regions - could also not be verified. However, some
aspects were not covered in this thesis, e.g. the distance of windows with
unusual counts to those containing the (effector) CDS.

The genome size alone cannot explain the observation of those windows. While
there is a trend that genomes with a larger total amount of analysed windows
have windows with unusual counts, there are cases like \textit{Phytophthora
castanetorum} with a total number of $n=6928$ analysed windows out of which
$295$ had unusual counts. In contrast to this, there exists \textit{Phytophthora
cinnamomi} with a total number of $n=10799$ analysed windows of which only $28$
have unusual counts. Also, by construction of the analysis, ambiguous bases
cannot explain the differences as all windows with a complexity score $C_m < 0$
where excluded. 

Computationally, it is interesting that the existing tools all use MurMur hash
to obtain hash values
\cite{ondovMashFastGenome2016,bagciMicrobialPhylogeneticContext2021,irberLightweightCompositionalAnalysis2022}.
In the last 10 years, many non cryptographic hash functions were published that
exceed MurMur hash performance, which in turn can be utilized for faster
runtime. It would be interesting to see if hash functions that are specialized
for hashing consecutive $k$-mers could be utilized to improve the performance in
that part of the programs even further, i.e \texttt{ntHash}
\cite{mohamadiNtHashRecursiveNucleotide2016}.

Altough the used hash functions are faster, the implementation of
\texttt{fmhdist} still only beats \texttt{sourmash} in terms of runtime when
exploting the fact that sourmash does not support multithreading out of the box.
When using a single core, both \texttt{sourmash} and \texttt{mash} outperform
\texttt{fmhdist} in terms of runtime. Without a deep dive into the architecture
of all involved implementations, one could only speculate about the reasons, the
choice of programming language for the heavy lifting (\texttt{rust} in the case
of \texttt{sourmash}, \texttt{c++} in the case of \texttt{mash}) could play a
role here. It is interesting, however, that \texttt{sourmash} is considerably
slower than \texttt{mash} using only a single thread. Using a scaling parameter
of $s=1000$ by default, most of the resulting sketches (median size is $54513$)
are considerably larger than the $10000$ used by \texttt{mash} in the
benchmarking setup. This obviously adds computational cost and could be among
the reasons why \texttt{sourmash} (and to some extend, \texttt{fmhdist}) is
slower than \texttt{mash} in the benchmark.

\cleardoublepage

\chapter{Conclusion}
In this thesis, I have implemented FracMinHash in a tool called \texttt{fmhdist}
to obtain distances and phylogenies for genomic sequences of
\textit{Phytophthora}. The results show that FracMinHash can in general be used
to reproduce published phylogenies in terms of clade clustering and, to some
extend, also in terms of clade placement. However, shorter sequences (such as
mtDNA) are too short, the resulting phylogenies don't depict clade relationships
correctly. FracMinHash sketches are typically larger compared to Mash sketches
and contain windows that are not as dense as one would expect. However, this
does not seem to have a direct impact on the biological interpreation of the
produced outlines.  
Altough \texttt{fmhdist} is slower than other implemantions such as
\texttt{mash} and \texttt{sourmash}, it is still very much usable on household
notebooks, enabling rapid sketch calculations and thus, experiments with
different parameters, hash functions and input sequences, if needed. An example
pipeline could start with preparing a reference database of relevant species
using \texttt{fmhdist db}, which is a matter of minutes, sketching the query
sequences in question with \texttt{fmhdist sketch}, calculating the distances to
only the closest genomes from the reference database with \texttt{fmhdist
ref-dist} and finally calclulating the phylogenetic outline using
\texttt{fmhdist outline}. Although the outline itself is also generated very
fast, the options for the user are somewhat limited. For complex outlines with
many taxa, one would probably want to apply complex formatting to highlight
several aspects of the outline. This should then be done using a tool that is
specialized on those capabilities such as SplitsTree 6
\cite{husonApplicationPhylogeneticNetworks2006}. Nevertheless, the resulting
outline can give a good overview of the phylogenetic neighborhood and can thus
play a part in setting the phylogenetic context, e.g. for \textit{Phytophthora}.