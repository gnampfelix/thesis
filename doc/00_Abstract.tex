% !TEX root = thesis.tex
%%%%%%%%%%%%%%%%%%%%%%%%%%%%%%%%%%%%%%%%%%%%%%%%%%%%%%%%%%%%%%%%%%%%
% Abstract, Zusammenfassung, Danksagung
%%%%%%%%%%%%%%%%%%%%%%%%%%%%%%%%%%%%%%%%%%%%%%%%%%%%%%%%%%%%%%%%%%%%

\section*{Abstract}
Many phylogenetic methods to communicate evolutionary relationships exist, one
of which is the phylogenetic outline
\cite{bagciMicrobialPhylogeneticContext2021}. This method is currently
implemented in tools such as \texttt{SplitsTree} 6
\cite{husonApplicationPhylogeneticNetworks2006} using Mash to obtain distances
for the input sequences \cite{ondovMashFastGenome2016}. Recently, FracMinHash
was published as a method to obtain evolutionary distances similar to Mash
\cite{irberLightweightCompositionalAnalysis2022}. In FracMinHash, all $k$-mers
$i$ of a sequence are hashed to obtain a value $h(i)$, all hashes below a given
threshold constitute a sketch. Two sketches can be compared using set
operations. This thesis implements distance estimation with FracMinHash in a
tool called \texttt{fmhdist} and uses this tool to explore the resulting
phylogenies for genomes of \textit{Phytophthora}. For closely related species,
clade membership and, to some extend, clade relationships are represented well.
Relationships of distantly related genomes, such as \textit{Phytophthora
cinnamomi} and fungal organisms found in soil samples of orchards infected by
\textit{Phytophthora cinnamomi}, are better represented by FracMinHash distances
than by Mash distances. FracMinHash cannot be applied on short genomic sequences
(e.g mtDNA) to reproduce phylogenies that are close to literature. For some
\textit{Phytophthora} genomes, FracMinHash sketches contain windows that have a
different number of hashes than expected, those windows can be linked to low
sequence complexity. \texttt{fmhdist} is slower than comparable tools, but still
is able to set the phylogenetic context
\cite{bagciMicrobialPhylogeneticContext2021} for a query sequence in a few
minutes. The source code is available at
\url{https://github.com/gnampfelix/thesis}.
\newpage


\section*{Acknowledgements}
I want to thank Prof. Huson and Prof. Lockhart for agreeing to review my thesis
and for their input during the project. Moreover, I want to thank Banu Cetinkaya
and Anupam Gautam for their input, support and feedback during all phases of the
thesis. Finally, I want to thank Niklas Agethen, Michael Mederer and Isabelle
Steinhauser for their feedback on the manuscript.