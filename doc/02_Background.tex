% !TEX root = thesis.tex
%%%%%%%%%%%%%%%%%%%%%%%%%%%%%%%%%%%%%%%%%%%%%%%%%%%%%%%%%%%%%%%%%%%%
% Introduction
%%%%%%%%%%%%%%%%%%%%%%%%%%%%%%%%%%%%%%%%%%%%%%%%%%%%%%%%%%%%%%%%%%%%

\chapter{Background}
  \label{sec:background}

\section{Phylogenetic research about \textit{Phytophthora}}


\section{Mash and Phylogenetic Outlines}
The phylogenies discussed above are expensive both in terms of labor and in
terms of computation: typical approaches include the prediction of conserved
genes, the alignement of the corresponding sequences and then the inference of a
phylogenetic, for example using Bayesian Inference
\cite{abadPhytophthoraTaxonomicPhylogenetic2023a,winkworthComparativeAnalysesComplete2022}.

Methods that utilize evolutionary distances
\cite{saitouNeighborjoiningMethodNew1987} are faster \todo{cite?} than Bayesian
inference but require some form of distance matrix as input. One could obtain
such a distance, e.g. by utilizing the average nucleotide identiy (ANI)
\cite{leeOrthoANIImprovedAlgorithm2016}. However, this in turn introduces a
computational cost and is not suited for distantly related species.

Another method, Mash, estimates the evolutionary distance by utilizing a concept
from web search engines from the early days of the internet: MinHash
\cite{broderResemblanceContainmentDocuments1998a,ondovMashFastGenome2016}. While
this method is about estimating the similarity between two different documents
on the world wide web, \Citeauthor{ondovMashFastGenome2016} applied the concept
to nucleotide sequences.

Such a sequence $A$ can be decomposed into a set containing all substrings of
length $k$, the so called $k$-mers, using $k(A)$. \todo{do I need a mathematical
definition here?} Using a \textit{hash function} $h: \Omega \rightarrow [0, H]$
with usually $H=2^{64}$ or $H=2^{128}$ on modern computers, one can obtain the
\textit{hash value} $h(i)$ for each such $k$-mer.

For the sequence $A$ the \textit{sketch} $S(A)$ is the set of the $s$
smallest $h(i) \forall i \in k(A)$. One can use the sketches of two genomes
$A$ and $B$ to estimate the Jaccard similarity with 

\begin{align}
  J(A, B) = \frac{|A \cap B|}{|A \cup B|} \approx \frac{|S(A \cup B) \cap S(A) \cap S(B)|}{|S(A) \cup S(B)|}
\end{align}

This similarity can then be used to obtain a evolutionary distance using

\begin{align}
  D_{Mash}(A,B) = -\frac{1}{k}\ln{\frac{2J(A,B)}{1+J(A,B)}}
\end{align}

This method is widely used to estimate the evolutionary distances. In the
original publication, the authors created Mash sketches for all bacterial
\todo{check wording here, is it bacterial?} genomes in NCBI, estimated the
distances based on those and calcualted an evolutionary tree
\cite{ondovMashFastGenome2016}. It is also incorportated into FastANI to
estimate ANI scores, well, fast \cite{jainHighThroughputANI2018}.

A third use case for Mash is the creation of phylogenetic context using
phylogenetic outlines \cite{bagciMicrobialPhylogeneticContext2021}.

\section{FracMinHash}