% !TEX root = thesis.tex
%%%%%%%%%%%%%%%%%%%%%%%%%%%%%%%%%%%%%%%%%%%%%%%%%%%%%%%%%%%%%%%%%%%%
% Introduction
%%%%%%%%%%%%%%%%%%%%%%%%%%%%%%%%%%%%%%%%%%%%%%%%%%%%%%%%%%%%%%%%%%%%

\chapter{Background}
  \label{sec:background}

For this thesis, several experiments and analysis were conducted involving Mash,
FracMinHash, phylogenetic trees and phylogenetic outlines using genomic
sequences of \textit{Phytophthora}. This chapter aims to introduce those topics
such that the experiments and analysis can be better understood.

\section{Phylogenetic trees and phylogenetic outlines}
Let $X$ be a set of $n$ taxa. A \textit{phylogenetic tree} on $X$ is a tree of
which the leaves are bijectively labeled by $X$. One way to calculate a
phylogenetic tree for $X$ using a distance matrix $D$ is Neighbor-Joining
\cite{saitouNeighborjoiningMethodNew1987}. An example tree can be seen in
Figure~\ref{fig:exampleTree}. 

To serialize a tree, one could use the Newick format
\cite{pavlopoulosReferenceGuideTree2010}. In this format, leaves are represented
by their label, siblings are separated by comma. A tree is put in to
parantheses, the same applies for subtrees. Two sibling subtrees are thus also
separated by comma. The format also supports the inclusion of edge lengths by
putting them after a double colon after the subtree or leaf. I am mostly
interested in tree topology, so I won't include edge lengths into any Newick
strings in this thesis. The tree depicted in Figure~\ref{fig:exampleTree}
serialized to Newick looks like this:

\texttt{((A, B), C)}

This tree tree also enables an intuitive
perspective at the concept of a \textit{split}: a split $S$ is a bipartition of
$X$. In the case of phylogenetic trees, each edge in the tree is thus a split.
Formally, a \textit{split} is a biparition $S=A|B$ with $A, B \subset X$
such that $A \neq \emptyset$, $B \neq \emptyset$, $A \cap B = \emptyset$ and $A
\cup B = X$.  \todo{citation?}

The \textit{weight} of a split $\omega(S)$ is also directly depicted in the tree
as the length of the edge. 

\begin{figure}
  \centering
    \begin{tikzpicture}[node distance={10mm}, minimum size={2mm}, inner
      sep={0pt}, thick, main/.style = {draw, circle, fill},
      color_edge/.style = {color=cyan}]
      \node[](l0){};
      \node[below = of l0](l1){};
      \node[below = of l1](l2){};
      \node[below = of l2](l3){};

      \node[main, right = 10mm of l0](root){};
      \node[main, right = 10mm of l1](abc){} edge(root);
      \node[main, right = 5mm of l2] (ba){} edge(abc);
      \node[right = 0mm of l3] (b) {B} edge(ba);
      \node[right = 10mm of l3](a) {A} edge(ba);
      \node[right = 20mm of l3] (c) {C} edge[color_edge](abc);
    \end{tikzpicture}
  \caption{Example of a rooted phylogenetic tree on the taxa $X = \{A, B, C\}$.
  The leaves are labeled with the taxa, the internal nodes don't have labels.
  Each edge can be seen as a split $S=A|B$, e.g. the cyan edge is the split $S =
  \{A, B\} | \{C\}$.}
  \label{fig:exampleTree}
\end{figure}

Phylogenetic trees are widely used to communicate evolutionary relationships
between species
\cite{mandalComparativeGenomeAnalysis2022,winkworthComparativeAnalysesComplete2022,ayala-usmaWholeGenomeDuplication2021}.
However, as trees do not allow loops, they are not able to display evolutionary
events such as horizontal gene transfer or hybridization. Those features are
supported by phylogenetic networks, which allow loops. \todo{maybe cite?} The
Neighbor-Net algorithm \cite{bryantNeighborNetAgglomerativeMethod2004} is one
algorithm to clauclate a phylogenetic network. 

This algorithm calculates the set of weighted splits $\Sigma$ using the distance
matrix $D$. Those splits could already be displayed as a valid phylogenetic
network, however this introduces many, many edges and internal nodes that are
hard to interpret. Phylogenetic outlines provide a simplification of that by
replacing large areas of edges and internal nodes with a single polygon that
only keeps the outermost edges and internal nodes.
\cite{bagciMicrobialPhylogeneticContext2021}. An example of such a phylogenetic
outline can be seen in Figure~\ref{fig:outlineExample}. 

\begin{figure}
  \centering
  \begin{tikzpicture}[node distance={10mm}, minimum size={2mm}, inner
    sep={0pt}, thick, main/.style = {draw, circle, fill},
    color_edge/.style = {color=cyan}]
    \node[](l0){};
    \node[below = of l0](l1){};
    \node[below = of l1](l2){};
    \node[below = of l2](l3){};
    \node[below = of l3](l4){};

    \node[main, right = 20mm of l0](root){};
    \node[main, right = 20mm of l1](i1){} edge(root);
    \node[main, right = 10mm of l2](i2){} edge[color_edge](i1);
    \node[main, right = 30mm of l2](i4){} edge(i1);
    \node[main, right = 20mm of l3](i3){} edge(i2) edge[color_edge](i4);

    \node[right = 0mm of l2](a) {A} edge(i2);
    \node[right = 20mm of l4](b) {B} edge(i3);
    \node[right = 40mm of l2](c) {C} edge(i4);
  \end{tikzpicture}
  \caption{Example of a rooted phylogenetic outline on the taxa $X = \{A, B,
  C\}$. The outer nodes are labeled with the taxa, the internal nodes don't have
  labels. In this example, each edge and each pair of parallel edges can be seen
  as a split, e.g. the cyan edges depict the split $S = \{A, B\} | \{C\}$}
  \label{fig:outlineExample}
\end{figure}

\section{Phylogenetic research on \textit{Phytophthora}} 

The first scientific publication on \textit{Phytophthora} was in 1876 by
Heinrich Anton de Bary \cite{kroonGenusPhytophthoraAnno2012} \todo{find citation
for original publication}. Since then, extensive research has been conducted on
a variety of aspects about this genus. Initially, it was thought to be part of
the fungi kingdom because of several morphological similarities
\todo{citation!}, but is now classified on the phylum rank as Oomycota, which is
a direct child of the Eukaryote superkingdom \todo{cite, check
https://bsppjournals.onlinelibrary.wiley.com/doi/10.1111/mpp.12568}.

- Currently ongoing Phytophthora plagues

There are many revisions of \textit{Phytophthora} phylogeny, e.g.
\cite{kroonGenusPhytophthoraAnno2012,yangExpandedPhylogenyGenus2017,abadPhytophthoraTaxonomicPhylogenetic2023a}.
Typically, the genus is divided into multiple clades, each with a numerical
identifier. The exact number and relationsips of the clades is different from
publication to publication. As I will use those clade topologies later, I will
quickly layout the topologies given by the three publications above in Newick
format.

\Citeauthor{kroonGenusPhytophthoraAnno2012} state the following topology:\\
\texttt{((((((((1, 4), 2), 5), 3), 6), 7), 8), (9, 10))}\cite{kroonGenusPhytophthoraAnno2012}

\Citeauthor{yangExpandedPhylogenyGenus2017} state the following topology:\\
\texttt{((((((((1, 4), 2), 3), 5), 7), 6), 8), (9, 10))}\cite{yangExpandedPhylogenyGenus2017}

\Citeauthor{abadPhytophthoraTaxonomicPhylogenetic2023a} state the following topolgy:\\
\texttt{(((((((((1, 2), (12, 4)), (3, 13)), 5), 6), 7), 11), 8), (9, 10))}\cite{abadPhytophthoraTaxonomicPhylogenetic2023a}

\todo{I think I might have a little mistake in my results when stating that the (8, 10) cluster is important.}

Given the rise of modern sequencing techniques, and thus the increased abundance
of available genomic data, more and more studies are also comparing different
aspects of \textit{Phytophthora} genomes, e.g. the location and content of
transposable elements and simple sequence repeats
\cite{mandalComparativeGenomeAnalysis2022}, the evolution of mitogenome
sequences involving other Peronosporaceae genomes
\cite{winkworthComparativeAnalysesComplete2022}, the diversity of
\textit{Phytophthora} genomes found in soil and water samples
\cite{catalaUseGenusSpecificAmplicon2015} or the differences in species
communities associated with soy beans
\cite{navarroComparisonSpeciesCommunities2021}.

The findings include that some \textit{Phytophthora} species are
hybrids, e.g. \textit{Phytophthora $\times$cambivora}
\cite{jungSixNewPhytophthora2017,vanpouckeUnravellingHybridizationPhytophthora2021}.
Given this information, the use of phylogeneitc networks over phylogenetic trees
is becoming interesting. 

Also of interest is the concept of the two-speed genome model. Under this
hypothesis, some \textit{Phytophthora} genomes have different evolutionary
rates, i.e. the regions containing the household genes evolve slower than those
regions containing the effector genes \cite{dongTwospeedGenomesFilamentous2015}.
The fast evolving regions are gene sparse and repeat rich, the latter seems to
enable faster evolution on a molecular level with repeat induced point mutations
\cite{dongTwospeedGenomesFilamentous2015}. There is evidence for this model for
various \textit{Phytophthora} species including \textit{Phytophthora cinnamomi}
\cite{engelbrechtGenomeDestructiveOomycete2021} and \textit{Phytophthora
infestans}\cite{ayala-usmaWholeGenomeDuplication2021,dongTwospeedGenomesFilamentous2015},
but there is also evidence that not all \textit{Phytophthora} species have this
genome architecture, e.g. \textit{Phytophthora betacei}
\cite{ayala-usmaWholeGenomeDuplication2021}

\textit{Phytophthora cinnamomi} and \textit{Phytophthora infestans} receive lots
of attention because of their impact on wildlife and agriculture:
\textit{Phytophthora cinnamomi} is reported to target over 5000 different hosts
such as avocado trees in Europe and natural vegeation in Australia
\cite{hardhamPhytophthoraCinnamomi2018,solis-garciaPhytophthoraRootRot2020},
\textit{Phytophthora infestans} is most famously known for its devestating
impact on potatoes and tomatoes \cite{ayala-usmaWholeGenomeDuplication2021}.


\section{Mash}
The phylogenies discussed above are expensive both in terms of labor and in
terms of computation: typical approaches include the prediction of conserved
genes, the alignement of the corresponding sequences and then the inference of a
phylogenetic, for example using Bayesian Inference
\cite{abadPhytophthoraTaxonomicPhylogenetic2023a,winkworthComparativeAnalysesComplete2022}.

Methods that utilize evolutionary distances
\cite{saitouNeighborjoiningMethodNew1987} are faster \todo{cite?} than Bayesian
inference but require some form of distance matrix as input. One could obtain
such a distance, e.g. by utilizing the average nucleotide identiy (ANI)
\cite{leeOrthoANIImprovedAlgorithm2016}. However, this in turn introduces
computational cost and is not suited for distantly related species.

Another method, Mash, estimates the evolutionary distance by utilizing a concept
from web search engines from the early days of the internet: MinHash
\cite{broderResemblanceContainmentDocuments1998a,ondovMashFastGenome2016}. While
this method is about estimating the similarity between two different documents
on the world wide web, \Citeauthor{ondovMashFastGenome2016} applied the concept
to nucleotide sequences.

Let $A = a_1 a_2 \dots a_l$ be a string with length $l$ on the DNA alphabet with
$a_i = \{A, T, G, C\}$. $A$ can be decomposed into a set containing all
substrings of length $k$, the so called $k$-mers, using $k(A) = \{a_i \dots
a_{i+k-1} | 1 \leq i < l-k-1\}$. Using a \textit{hash function} $h: \Omega
\rightarrow [0, H]$ with typically $H=2^{64}$ or $H=2^{128}$ on modern
computers, one can obtain the \textit{hash value} $h(i)$ for each such $k$-mer.

For the sequence $A$ the \textit{Mash sketch} $S(A)$ is the set of the
$s_{mash}$ smallest $h(i) ~ \forall i \in k(A)$. Note that the parameter
$s_{mash}$ is called $s$ in the original Mash publication
\cite{ondovMashFastGenome2016}, but as this parameter is also used in
FracMinHash with different semantics, I denote it as $s_{mash}$. It was shown
that those sketches can be used as a sample for the input sequence to answer
questions of similarity \cite{ondovMashFastGenome2016}. One can use the sketches
of two genomes $A$ and $B$ to estimate the Jaccard similarity with 

\begin{align}
  J(A, B) = \frac{|A \cap B|}{|A \cup B|} \approx \frac{|S(A \cup B) \cap S(A) \cap S(B)|}{|S(A) \cup S(B)|}
\end{align}

This similarity can then be used to obtain a evolutionary distance using

\begin{align}
  D_{Mash}(A,B) = -\frac{1}{k}\ln{\frac{2J(A,B)}{1+J(A,B)}}
\end{align}

This method is widely used to estimate the evolutionary distances. In the
original publication, the authors created Mash sketches for all bacterial
\todo{check wording here, is it bacterial?} genomes in NCBI, estimated the
distances based on those and calcualted an evolutionary tree
\cite{ondovMashFastGenome2016}. It is also incorportated into FastANI to
estimate ANI scores, well, fast \cite{jainHighThroughputANI2018}.

A third use case for Mash is the creation of phylogenetic context using
phylogenetic outlines \cite{bagciMicrobialPhylogeneticContext2021} in
metagenomic studies. Here, one sequences all DNA found in a given sample, e.g.
soil or sea water, and assembles (draft) genomes from the acquired reads
\cite{kuninBioinformaticianGuideMetagenomics2008}. Outlines help with
identifying the species to which the draft genome belongs. For this, the
phylogenetic context is established by placing the draft genome in a
phylogenetic outline using all sequences from a reference databases that have a
Mash distance to the query below a user defined threshold
\cite{bagciMicrobialPhylogeneticContext2021}.

Mash itself is implemented in a tool called \texttt{mash}
\cite{ondovMashFastGenome2016}. For convenience, there exists also a tool called
\texttt{mashtree} \cite{katzMashtreeRapidComparison2019} which utilized Mash to
directly calculate distances and a phylogenetic tree given some input sequences.

\section{FracMinHash}
Mash is not the only method in the area of \textit{locality sensitive hashing}.
A recently published method is called FracMinHash by
\Citeauthor{irberLightweightCompositionalAnalysis2022}
\cite{irberLightweightCompositionalAnalysis2022} and aims to tackle some of the
issues with Mash, namely the fact that Mash is not optimal for containment
estimation and that it is not optimal for distance estimation for genomes with
different sizes.

The key idea of FracMinHash is to define a threshold $\frac{H}{s}$ with $H$
being the largest value that $h$ can produce and $s$ a user defined scaling
parameter with $0 < s \leq H$ such that a FracMinHash sketch only contains hash
values below that threshold. As shown in Section~\ref{sec:res}, the sketch size can be
approximated by $\frac{n}{s}$ where $n$ is the size of the input sequence
\cite{irberLightweightCompositionalAnalysis2022,heraDebiasingFracMinHashDeriving2023}.


Formally, the FracMinHash sketch is defined as 

\begin{align}
  \mathbf{FRAC}_s(A) = \{h(i) \leq \frac{H}{s} | \forall i \in k(A)\}
\end{align}

Where a Mash sketch keeps the $s_{mash}$ smallest hashes, FracMinHash keeps all
those hashes that are below a given threshold $\frac{H}{s}$

This approach has the advantage that the sketching can be applied to streams: it
is known if a hash is part of the sketchas soon as the $k$-mer is hashed. This
also applies when calculating the intersection between two sketches. 

In Mash, to estimate the Jaccard index, it is required to also find the
$s_{mash}$ smallest overall hashes of the two input sequences, i.e. $S(A \cup
B)$. This is not needed for FracMinHash, as all hashes that are part of the
input sketches already satisfy the condition $\leq \frac{H}{s}$. Thus, the
Jaccard estimation including a factor correcting for bias is given as

\begin{align}
  J_{frac} = \frac{\hat{J}_{frac}}{1 - (1 - s)^{|A \cup B|}}
\end{align}

with

\begin{align}
  \hat{J}_{frac}(A, B) = \frac{|\mathbf{FRAC}_s(A) \cap \mathbf{FRAC}_s(B)|}{|\mathbf{FRAC}_s(A) \cup \mathbf{FRAC}_s(B)|}  
\end{align}

This Jaccard index can be used to calculate evolutionary distance, for which the
FracMinHash publications follow again a different approach from Mash. Mash uses
a Poisson model that assumes that all $k$-mers mutate independently
\cite{ondovMashFastGenome2016,heraDebiasingFracMinHashDeriving2023,fanAssemblyAlignmentfreeMethod2015},
whereas FracMinHash assumes a simple model in which each nucleotide $a_i$ of a
sequence $A$ mutates at a fixed rate $p$
\cite{heraDebiasingFracMinHashDeriving2023}. The authors note that such a
mutated sequence $A'$ has an average nucleotide identity (ANI) of $1-p$ to $A$.
As the inverse can be used to calculate a distance from any similarity ($D = 1 -
I$), I will denote $p$ in the following as $D_{frac}$. Following this, the
authors define an estimation of the distance 

\begin{align}
  D_{frac}(A, B) = 1 - (\frac{2J_{frac}(A,B)}{1+J_{frac}(A, B)})^{\frac{1}{k}}
\end{align}

FracMinHash also defines a containment index and the corresponding distance:
\begin{align}
  C_{frac}(A, B) = \frac{|\mathbf{FRAC}_S(A) \cap \mathbf{FRAC}_s(B)|}{|\mathbf{FRAC}_S(A)| (1-(1-s)^{|A|})}
\end{align}
\begin{align}
  D=1-C_{frac}^{\frac{1}{k}}
\end{align}

FracMinHash is implemented in a tool called \texttt{sourmash}
\cite{irberLightweightCompositionalAnalysis2022,irberDecentralizingIndicesGenomic2020}.
For the purpose of this thesis, I have implemented the method in
\texttt{fmhdist} to obtain phylogenetic outlines as described in
\cite{bagciMicrobialPhylogeneticContext2021}.
