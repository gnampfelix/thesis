\documentclass[aspectratio=169]{beamer}
\usetheme[progressbar=foot,numbering=fraction]{metropolis} % Use metropolis theme

\usepackage{graphicx}
\usepackage{booktabs}

\usepackage[backend=biber,
            style=authortitle,
            maxnames=5,
            isbn=false,
            sortcites=true,
            hyperref=false,
            url=false]{biblatex}
\addbibresource{mylit.bib}
\setbeamerfont{footnote}{size=\scriptsize}
\title{Application of FracMinHash to analyse the phylogenetic context of \textit{Phytophthora}}

\date{\today}
\author{ Felix Seidel \\
\href{mailto:felix.seidel@student.uni-tuebingen.de}{felix.seidel@student.uni-tuebingen.de}
}
\begin{document}

\maketitle

\begin{frame}{Overview}
    \tableofcontents
\end{frame}

\section{A short story}
\begin{frame}{The setting}
    Picture of idyllic farms, potato, maybe avocado?
\end{frame}

\begin{frame}{The villain: \textit{Phytophthora}}
    Picture of affected plants, maybe three bullet points (yearly damage)
\end{frame}

\begin{frame}{The hero?}
    "Unfortunately, we cannot just walk to some mountain and throw jewelrey in
    it to defeat the villain" Research on the modes of operation, effector
    genes, and \textbf{phylogeny} of that species
\end{frame}

\section{Background}
\begin{frame}{Phylogenetic Tree}

\end{frame}

\begin{frame}{Phylogenetic Outline \footcite{bryantNeighborNetImprovedAlgorithms2023,bagciMicrobialPhylogeneticContext2021}}
    
\end{frame}

\begin{frame}{Phylogenetic Context \footcite{bagciMicrobialPhylogeneticContext2021}}
    
\end{frame}

\begin{frame}{FracMinHash \footcite{irberLightweightCompositionalAnalysis2022}}
    I think I skip Mash, if I need more time I can add it. Need to ensure that
    this slide is to get some distance estimation that we can use to calculate
    the outlines
\end{frame}

\section{Phylogenetic Context of \textit{Phytophthora}}
\begin{frame}{The implementation with \texttt{fmhdist}}
    Maybe some block diagram to illustrate the pipeline?
\end{frame}

\begin{frame}{An outline produced by \texttt{fmhdist}} 
    don't provide slide that
    discusses the big outlines, boring and not much value. mention it, but don't discuss this.
\end{frame}

\begin{frame}{Compare this to Mash}
    maybe put both outlines on one slide?
\end{frame}

\begin{frame}{Origin of the hashes in the sketch (1)}
    Show plot that I've used in the thesis, maybe without the complexity, then
    with the complexity
\end{frame}

\begin{frame}{Origin of the hashes in the sketch (2)}
    Show the plot for P. infestans
\end{frame}

\begin{frame}{Statistical analysis}
    some genomes have those windows, some don't
\end{frame}

\begin{frame}{Benchmarks}

\end{frame}

\section{Conclusion}
\begin{frame}{Method works in general}

\end{frame}

\begin{frame}{Potential next steps}
    
\end{frame}

\begin{frame}[standout]
    Thanks!
\end{frame}
\end{document}